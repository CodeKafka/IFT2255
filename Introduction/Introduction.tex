\documentclass[8pt]{report}
\input{preamble}

\title{\Huge{Cours 1 Introduction}\\}
\author{\huge{Franz Girardin}}
\date{5 Septembre 2023}
\lstset{inputencoding=utf8/latin1}

\usepackage{balance}
\usepackage{lmodern}
\usepackage{dirtree}
\usepackage{titlesec}
\titleformat{\chapter}
  {\small\bfseries} % format
  {}                % label
  {0pt}             % sep
  {\huge}           % before-code


\usepackage{lipsum}
\usepackage{titling}
\renewcommand\maketitlehooka{\null\mbox{}\vfill}
\renewcommand\maketitlehookd{\vfill\null}





\usepackage{afterpage}
\newcommand\myemptypage{
    \null
    \thispagestyle{empty}
    \addtocounter{page}{-1}
    \newpage
    }

%====================================================================

%====================================================================
\begin{document}
\maketitle
\pagebreak
\tableofcontents
\pagebreak

\chapter{Introduction}
\section*{Définitions simples}

\begin{Definition*}{Génie Logiciel}{}
    Branche de l'ingénierie associée au développment de logiciels
    utilisant des principes, méthode et procédures définis. 
\end{Definition*}

\begin{Definition}{Logiciel}{}
    Collection de code exécutable, ainsi que de bibliothèques 
    documentation associés.  
\end{Definition}
Il est possible 
\end{document}

